\section{Introducción}

\subsection{Propósito}

\textbf{Propósito del Documento:} Definir de forma precisa, completa y no ambigua el comportamiento esperado, los límites operacionales y los criterios de aceptación del sistema AllConnect Market. Este documento sirve como base contractual para:

\begin{itemize}
    \item Habilitar el diseño arquitectónico detallado (SAD)
    \item Guiar el análisis de arquitectura (ATAM)
    \item Establecer criterios de verificación y validación
    \item Facilitar la comunicación entre stakeholders
    \item Servir como referencia durante todo el ciclo de vida del proyecto
\end{itemize}

\subsection{Objetivos de Negocio}

Los objetivos de negocio del sistema AllConnect Market se fundamentan en cinco pilares estratégicos definidos en el modelo de negocio:

\subsubsection{Pilar 1: Consolidación Multicanal mediante Identidad y Perfil Unificado}

\textbf{Objetivo:} Crear una experiencia de usuario única donde el cliente gestiona todas sus compras, suscripciones y servicios desde una única cuenta.

\textbf{Valor de Negocio:}
\begin{itemize}
    \item Reducción de fricción en el proceso de compra
    \item Incremento en la retención de usuarios (target: +40\% en primer año)
    \item Datos consolidados para personalización efectiva
    \item Ventaja competitiva vs. plataformas especializadas verticalmente
\end{itemize}

\textbf{Habilitadores Técnicos:}
\begin{itemize}
    \item Sistema de autenticación único (SSO) con MFA opcional
    \item Perfil de usuario unificado con preferencias cross-vertical
    \item Historial consolidado de transacciones
    \item Motor de recomendaciones basado en comportamiento multi-vertical
\end{itemize}

\subsubsection{Pilar 2: Carrito de Compras Universal}

\textbf{Objetivo:} Implementar la capacidad técnica para que el usuario pueda añadir bienes físicos, suscripciones digitales y servicios profesionales en una misma transacción.

\textbf{Valor de Negocio:}
\begin{itemize}
    \item Incremento del valor promedio del carrito (AOV) estimado en +35\%
    \item Reducción del cart abandonment rate (target: $<$25\%)
    \item Simplificación del proceso de checkout
    \item Diferenciación clara en el mercado
\end{itemize}

\textbf{Complejidad Gestionada:}
\begin{itemize}
    \item Lógica de fulfillment diferenciada por tipo de producto
    \item Cálculo de impuestos y comisiones por vertical
    \item Gestión de estados heterogéneos (entrega, licencia, cita)
    \item Transacciones distribuidas con garantía de consistencia
\end{itemize}

\subsubsection{Pilar 3: Confianza y Trazabilidad mediante Pasarela de Pagos Centralizada}

\textbf{Objetivo:} Establecer un único punto de procesamiento de pagos seguro y eficiente que maneje transacciones puntuales y recurrentes.

\textbf{Valor de Negocio:}
\begin{itemize}
    \item Cumplimiento PCI-DSS nivel 1
    \item Payment success rate objetivo: $>$99\%
    \item Reducción de disputas mediante trazabilidad completa
    \item Habilitación de múltiples métodos de pago (tarjetas, wallets, BNPL)
\end{itemize}

\textbf{Capacidades Requeridas:}
\begin{itemize}
    \item Integración multi-pasarela con failover automático
    \item Tokenización de datos sensibles
    \item Gestión de pagos recurrentes para suscripciones
    \item Reconciliación y liquidación automatizada
\end{itemize}

\subsubsection{Pilar 4: Integración API-First con Proveedores}

\textbf{Objetivo:} Establecer una filosofía de integración profunda y automatizada que elimine la necesidad de portales de gestión manual.

\textbf{Valor de Negocio:}
\begin{itemize}
    \item Time-to-market de onboarding: $<$5 días para proveedores ancla
    \item Escalabilidad ilimitada de proveedores
    \item Sincronización en tiempo real de catálogos e inventarios
    \item Reducción de errores operativos por automatización
\end{itemize}

\textbf{Modelo Operativo:}
\begin{itemize}
    \item Sincronización catálogo/stock vía API RESTful
    \item Distribución automática de órdenes mediante webhooks
    \item Agendamiento de servicios integrado con calendarios de proveedores
    \item Developer Portal con sandbox y documentación OpenAPI
\end{itemize}

\subsubsection{Pilar 5: Analítica Operativa mediante Infraestructura de Notificaciones Multicanal}

\textbf{Objetivo:} Mantener informados a clientes y proveedores en cada etapa del proceso mientras se capturan métricas para decisiones de producto.

\textbf{Valor de Negocio:}
\begin{itemize}
    \item Reducción de tickets de soporte en 45\%
    \item Incremento en satisfacción del cliente (NPS target: $>$70)
    \item Datos accionables para optimización del funnel
    \item Visibilidad de KPIs en tiempo real (GMV, conversión, disputas)
\end{itemize}

\textbf{Canales Implementados:}
\begin{itemize}
    \item Email transaccional
    \item SMS para eventos críticos
    \item Push notifications en aplicación móvil
    \item Webhooks para integraciones de proveedores
\end{itemize}

\subsection{Alineación con Restricciones}

\textbf{Nota Importante:} Este SRS describe el alcance ideal del sistema propuesto, fundamentado en la estrategia de lanzamiento integral de los tres verticales definida en el modelo de negocio. Las restricciones de tiempo y presupuesto se gestionarán en la fase de implementación y quedarán reflejadas en:

\begin{itemize}
    \item \textbf{Plan de Entregas del SAD:} Priorización de MVPs por vertical
    \item \textbf{Roadmap de Producto:} Fases de rollout y features incrementales
    \item \textbf{Estrategia de Go-to-Market:} Segmentación de lanzamiento por región/mercado
\end{itemize}
